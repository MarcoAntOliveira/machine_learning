\documentclass{article}
\usepackage[legalpaper, left=1 cm, right=1cm, top=0.5cm, bottom=0.5cm] {geometry}
\date{} % Remove a exibição da data
\usepackage{xcolor}
\usepackage{listings}
\usepackage{graphicx}
\usepackage{hyperref} % Para criar links
\usepackage[utf8]{inputenc}
\usepackage[T1]{fontenc}

\lstdefinestyle{pythonStyle}{
    language=Python,
    basicstyle=\ttfamily\small,
    keywordstyle=\color{blue},
    stringstyle=\color{red},
    commentstyle=\color{blue}\itshape,
    numbers=left,
    numberstyle=\tiny\color{blue},
    breaklines=true,
    showstringspaces=false,
}



\title{Modulo 7}
\begin{document}
\maketitle
\tableofcontents
\newpage

\section{Conceitos fundamentais de machine learning e inteligencia artifical}
\textbf{Em quais aréas são usados as machines learning}
\begin{itemize}
    \item Aréas medicas
    \item Algoritmos de economias em empresa
    \item Navegacao inteligente
    \item recomendação de serviços e produtos
    \item Mercado financeiro
    \item marketing digital
\end{itemize}

\section{Machine Learning x Inteligencia artificial}
\textbf{Machine Learning é estudo de algoritmos computacionais que aprendem padrões automaticamente  através do uso de dados}

\begin{itemize}
    \item Artificial inteligence  A technique which enables machines to mimic human behavior
    \item Machine learning Subset of AI technique which use statistical methods to enable machines to improve with experience
    \item Deep Learning Subset of ML which make the computation of multi layer neural network feaside
\end{itemize}
\end{document}

\section{Estrutura de um projeto ML}
\textbf{modelos de machine learning}
\begin{itemize}
    \item decision trees
    \item KNN
    \item Randon forests
    \item k-means
    \item Logistic regression
\end{itemize}

\section{Tipos de algoritmos de ML}
\textbf{tipos de aprendizado}
\begin{itemize}
    \item Apredizado supervisonado \\ toma como conjunto amostral $70\%$ do regra de bolso
    \item Aprendizado não supervisionado
    \item Aprendizado por reforço
\end{itemize}

\section{Regressão x classificação}
\begin{itemize}
    \item Modelos de Regressão
    \item classificação
    \item politica de tomada de decisão
    
\end{itemize}

\section{Desafios do machine learning}
\begin{itemize}
    \item quantidade insuficiente de dados
    \item Dados de treinamento não repesentativos
    \item baixa qualidade de dados
    \item caracteristicas irrelevantes
    \item overlifting x underlifting
\end{itemize}

\section{Checklist do ML}
\begin{itemize}
    \item Olhar geral sobre oproblema
    \item obtenha os dados
    \item Vizualize e explore o conjunto de dados para gerar insights 
    \item prepare os dados para os algoritmos
    \item selecione um modelo e treine-o
    \item Otimize seu modelo
    \item Apresente sua solução 
    \item Coloque seu modelo em produção e mantenha seu sistema
    
\end{itemize}
